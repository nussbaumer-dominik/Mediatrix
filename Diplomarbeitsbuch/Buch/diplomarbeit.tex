\documentclass[
    headings=optiontotocandhead,% Erweiterung für das optionale Argument der
                                % Gliederungsbefehle aktiviert.
    twoside,
    numbers=noenddot,% Keine Punkte am Ende der Gliederungsnummern und davon
                     % abgeleiteten Nummern
    toc=flat, %Flache TOC
    12pt, % Schriftgröße
    titlepage, % es wird eine Titelseite verwendet
    parskip=full, % Abstand zwischen Absätzen (ganze Zeile)
    listof=totoc, % Verzeichnisse im Inhaltsverzeichnis aufführen
    listof=flat, % mehr Abstand für grosse Zahlen
    numbers=noenddot, % kein Punkt am Ende bei Nummern
    %%enlargefirstpage,% Gibt es bei scrartcl nicht!!!!
    bibliography=totoc, % Literaturverzeichnis im Inhaltsverzeichnis aufführen
    %index=totoc, % Index im Inhaltsverzeichnis aufführen
    %captions=tableheading, % Beschriftung von Tabellen für Ausgabe oberhalb
                           % der Tabelle formatieren
    %draft % Status des Dokuments (final/draft) draft hinzufügen zum anziegen
    %%der zeilen ende
    a4paper,DIV=14,
    BCOR=15mm,
    % captions=tablesignature,
]{scrbook}

\setcounter{secnumdepth}{3}

\usepackage{textcomp}

\usepackage[T1]{fontenc}
\usepackage[utf8]{inputenc}

\usepackage[english, ngerman, naustrian]{babel} % your native language must be the last one!!

\usepackage{lastpage}
\usepackage{listings}
\usepackage{blindtext}

%% Aufzählungen nicht so weit einrücken
\usepackage[inline]{enumitem}
%\setitemize{leftmargin=*}
% Listen etwas wenige einrücken, erfordert enumitem
\setitemize{leftmargin=*}

\usepackage{lmodern}

\usepackage{xspace}

\usepackage{graphicx}

%%? \usepackage{textcomp}
\usepackage[hyphens]{url}
\usepackage{makeidx}
\makeindex
%%? \usepackage{graphicx}
\usepackage[numbers]{natbib}
\PassOptionsToPackage{normalem}{ulem}
\usepackage{ulem}

\usepackage{needspace}

\setlength\partopsep{0.5ex}%schoenere Listen
\usepackage[bottom]{footmisc}%fussnote ganz unten

\usepackage[]{microtype}
\UseMicrotypeSet[protrusion]{basicmath} % disable protrusion for tt fonts

\usepackage{multirow}   % Allows table elements to span several rows.
\usepackage{booktabs}   % Improves the typesettings of tables.
\usepackage{subcaption} % Allows the use of subfigures and enables their referencing.
\usepackage[ruled,linesnumbered,algochapter]{algorithm2e} % Enables the writing of pseudo code.
\usepackage[usenames,dvipsnames,table]{xcolor} % Allows the definition and use of colors. This package has to be included before tikz.
\usepackage{nag}       % Issues warnings when best practices in writing LaTeX documents are violated.
\usepackage{todonotes} % Provides tooltip-like todo notes.

\usepackage{color}

%% bessere Suche im PDF
\input{glyphtounicode}
\pdfgentounicode=1
%%%%%%%%%%%%%%%%%%%%%%%%%%%%%%%%%%%%%%%%%%%%%%%%%%%%%%%%%%%%%%%%%%%%%%%%%%%%%%%%%%

%  Kopf und Fußzeilen -- links und rechts verschieden
\newcommand{\kopfseitenummer}{{\bfseries \thepage}}
\newcommand{\kopfkapl}{{\bfseries\leftmark}}
\newcommand{\kopfkapr}{{\bfseries\rightmark}}
\newcommand{\kopfbild}{\voffset7mm\includegraphics[width=25mm]{HTL3RLogoRGB}}
\newcommand{\kopfHTL}{Höhere Technische Bundeslehranstalt Wien 3, \\Rennweg 	Abteilung für Informationstechnologie}

\usepackage[automark,headsepline,footsepline,plainfootsepline]{scrlayer-scrpage}
%\automark[chapter]{chapter}% Eventuell wenn doppelseitig
\setkomafont{pageheadfoot}{\normalcolor\footnotesize\scshape}
\setkomafont{pagenumber}{\normalfont\normalsize}
\clearpairofpagestyles
\ihead{\headmark}
\ohead{\kopfbild}
\ifoot{\kapitelautor}
\ofoot{\pagemark}
\ModifyLayer[addvoffset=-.6ex]{scrheadings.foot.above.line}% Linie verschieben
\ModifyLayer[addvoffset=-.6ex]{plain.scrheadings.foot.above.line}% Linie verschieben
\setlength{\headheight}{32pt}

% alle Seiten mit Kopfzeile
\renewcommand{\chapterpagestyle}{scrheadings}

%% Kapitel - aufwändige Kapitelüberschriften
%Options: Sonny, Lenny, Glenn, Conny, Rejne, Bjarne, Bjornstrup
%\usepackage[Bjornstrup]{fncychap}
% Alternative:
%\usepackage{titlesec}

% Verzeichnisse - aufwändiger
%\usepackage{tocloft}


%% Code Beispiele
%% eine Variante
\usepackage{listings}
\renewcommand{\lstlistingname}{\inputencoding{utf8}Listing}
%% andere Variante
%\usepackage{minted}
%\setminted{
%  linenos,
%  frame=lines,
%  framesep=2mm,
%  breaklines=true
%}
% Beispiel
%\begin{listing}[H]
%\begin{minted}{bash}
%...
%\end{minted}
%\caption{Beschreibung}
%\end{listing}
%% dritte Variante
% mit/für pandoc
\input{text/00_pandoclisting.tex}

%% should be last packages
\usepackage{scrhack}

\usepackage[unicode=true,
 bookmarks=true,bookmarksnumbered=false,bookmarksopen=false,
 breaklinks=true,pdfborder={0 0 0},backref=false,colorlinks=false]
 {hyperref}
\hypersetup{pdftitle={Diplomarbeit Titel},
 pdfauthor={Wer auch immer},
 pdfsubject={Diplomarbeit},
 pdfkeywords={dies, das}}
\urlstyle{same} % don't use monospace font for urls

%% for pandoc
\providecommand{\tightlist}{%
  \setlength{\itemsep}{0pt}\setlength{\parskip}{0pt}}

% Auch Fußnoten bündig ausrichten
\deffootnote[]{1em}{1em}{\textsuperscript{\thefootnotemark\ }}
%% setup
\sloppy % weniger Meldungen
\voffset7mm % etwas nach unten

%% schöner: 10000 -- gar keine, 1000 als Mittelweg
\clubpenalty = 1000 % Schusterjungen verhindern
\widowpenalty = 1000 % Hurenkinder verhindern
\displaywidowpenalty = 1000

%%%%%%%%%%%%%%%%%%%%%%%%%%%%%%%%%%%%%%%%%%%%%%%%%%%%%%%%%%%%%%%%%%%%%%%%%%%%%%%%%%
\begin{document}
%% wir schreiben keine Umlaut mit "a "o
\shorthandoff{"}
%% mit kapitelautor kann man den Autor festlegen oder auf leer setzen - steht dann in der Fußzeile.
\newcommand{\kapitelautor}{}

\input{text/00_latex_shortcuts.tex}
%%%%%Anfang Titelseite
%\pagenumbering{roman}
\frontmatter % Switches to roman numbering
\title{Diplomarbeit}
\begin{titlepage}
\begin{minipage}[b]{1\columnwidth}
\parbox[b]{50mm}{\includegraphics[width=45mm]{HTL3RLogoRGB}}
\hfill
\parbox[b]{130mm}{\footnotesize \textsc{Höhere Technische Bundeslehranstalt} Wien 3, Rennweg\\
IT \& Mechatronik\\
\\
HTL Rennweg :: Rennweg 89b\\
A-1030 Wien :: Tel +43 1 24215-10 :: Fax DW 18
}\\
\mbox{}
\end{minipage}

\vspace{1cm}


\begin{center}
\textbf{\LARGE{}Diplomarbeit}{\large{}}\\
{\large{}\vspace{15mm}
 }\textbf{\large{}Mediatrix}\\
\textbf{\large{}Ausgeschriebener Titel der Diplomarbeit}\\
 \vspace{15mm}
 ausgeführt an der\\
 Höheren Abteilung für Informationstechnologie/Medientechnik\\
 der Höheren Technischen Lehranstalt Wien 3 Rennweg\\
 \vspace{1cm}
 im Schuljahr 2017/2018\\
 \vspace{1cm}
 durch\\
 \vspace{0.5cm}
\textbf{\large{}Nußbaumer Dominik}\\
\textbf{\large{}Scharwitzl Clemens}\\
\textbf{\large{}Steiner Florian}\\

\par\end{center}{\large \par}

\begin{center}
\vspace{20mm}
 \normalsize unter der Anleitung von\\
 \vspace{0.5cm}
 Fink Andreas\\
Stimpfl Franz
\par\end{center}

\begin{center}
\vspace{5mm}
Wien, \today
\par\end{center}

\end{titlepage}%%%%%%%%%%%%%%%%%%%%% Ende Titelseite %%%%%%%%%%%%%%%%%%%%%%

\chapter*{Kurzfassung}

Darum geht es.

\blindtext[1]


\chapter*{Abstract}
\selectlanguage{english}
\input{text/00_abstract.tex}
\selectlanguage{naustrian}

\chapter*{Ehrenwörtliche Erklärung}
\input{text/00_erklaerung.tex}

%%%%%%%%%%%%%%%%%%%%%%%%%%%%%%%%%%%%%%%%%%%%%%%%%%%%%%%%%%%%%%%%%%%%%%%%%%%%%%%%%%%%%%%%
\cleardoublepage{}
\tableofcontents{}
\cleardoublepage{}
\listoftables
\cleardoublepage{}
\listoffigures

%hier geht es los mit dem Text - auf einer rechten Seite
\cleardoublepage{}
%\pagenumbering{arabic}
\mainmatter


%%%%%%%%%%%%%%%%%%%%%% Chapter Einleitung %%%%%%%%%%%%%%%%%%%%%
\chapter{Einleitung}\label{Einleitung}

\section{Problemstellung}\label{Problemstellung}

\renewcommand{\kapitelautor}{Autor: Clemens Scharwitzl}

% Das ist der Text in meinem

\#Chapter\#

\begin{itemize}
\tightlist
\item
  Liste \emph{Liste }Liste *Liste
\end{itemize}


\section{Ziel der Arbeit}\label{Ziel-der-Arbeit}

\renewcommand{\kapitelautor}{Autor: Clemens Scharwitzl}

% Das ist der Text in meinem

\#Chapter\#

\begin{itemize}
\tightlist
\item
  Liste \emph{Liste }Liste *Liste
\end{itemize}


\section{Abgrenzung und Voraussetzungen}\label{Abgrenzung-und-Voraussetzungen}

\renewcommand{\kapitelautor}{Autor: Clemens Scharwitzl}

% Das ist der Text in meinem

\#Chapter\#

\begin{itemize}
\tightlist
\item
  Liste \emph{Liste }Liste *Liste
\end{itemize}


\section{Aufbau}\label{Aufbau}

\renewcommand{\kapitelautor}{Autor: Clemens Scharwitzl}

% Das ist der Text in meinem

\#Chapter\#

\begin{itemize}
\tightlist
\item
  Liste \emph{Liste }Liste *Liste
\end{itemize}


%%%%%%%%%%%%%%%%%%%%%% Chapter Hardware %%%%%%%%%%%%%%%%%%%%%
\chapter{Hardware}\label{Hardware}

\section{Gehäuse}\label{Gehäuse}

\renewcommand{\kapitelautor}{Autor: Clemens Scharwitzl}

% Das ist der Text in meinem

\#Chapter\#

\begin{itemize}
\tightlist
\item
  Liste \emph{Liste }Liste *Liste
\end{itemize}


\section{Raspberry Pi}\label{Raspberry-Pi}

\renewcommand{\kapitelautor}{Autor: Clemens Scharwitzl}

% Das ist der Text in meinem

\#Chapter\#

\begin{itemize}
\tightlist
\item
  Liste \emph{Liste }Liste *Liste
\end{itemize}


\section{Ein- und Ausschlatversögerung}\label{Ein-und-Ausschlatversögerung}

\renewcommand{\kapitelautor}{Autor: Clemens Scharwitzl}

% Das ist der Text in meinem

\#Chapter\#

\begin{itemize}
\tightlist
\item
  Liste \emph{Liste }Liste *Liste
\end{itemize}


\section{Lautsprecherschutzschaltung}\label{Lautsprecherschutzschaltung}

\renewcommand{\kapitelautor}{Autor: Clemens Scharwitzl}

% Das ist der Text in meinem

\#Chapter\#

\begin{itemize}
\tightlist
\item
  Liste \emph{Liste }Liste *Liste
\end{itemize}


\section{Infrarotsender}\label{Infrarotsender}

\renewcommand{\kapitelautor}{Autor: Clemens Scharwitzl}

% Das ist der Text in meinem

\#Chapter\#

\begin{itemize}
\tightlist
\item
  Liste \emph{Liste }Liste *Liste
\end{itemize}


\section{DMX Interface}\label{DMX-Interface}

\renewcommand{\kapitelautor}{Autor: Clemens Scharwitzl}

% Das ist der Text in meinem

\#Chapter\#

\begin{itemize}
\tightlist
\item
  Liste \emph{Liste }Liste *Liste
\end{itemize}


\section{Verkabelung}\label{Verkabelung}

\renewcommand{\kapitelautor}{Autor: Clemens Scharwitzl}

% Das ist der Text in meinem

\#Chapter\#

\begin{itemize}
\tightlist
\item
  Liste \emph{Liste }Liste *Liste
\end{itemize}


\section{Gehäusebelüftung}\label{Gehäusebelüftung}

\renewcommand{\kapitelautor}{Autor: Clemens Scharwitzl}

% Das ist der Text in meinem

\#Chapter\#

\begin{itemize}
\tightlist
\item
  Liste \emph{Liste }Liste *Liste
\end{itemize}


\section{Stromversorgung}\label{Stromversorgung}

\renewcommand{\kapitelautor}{Autor: Clemens Scharwitzl}

% Das ist der Text in meinem

\#Chapter\#

\begin{itemize}
\tightlist
\item
  Liste \emph{Liste }Liste *Liste
\end{itemize}


\section{Anschlüsse für den Anwender}\label{Anschlüsse-für-den-Anwender}

\renewcommand{\kapitelautor}{Autor: Clemens Scharwitzl}

% Das ist der Text in meinem

\#Chapter\#

\begin{itemize}
\tightlist
\item
  Liste \emph{Liste }Liste *Liste
\end{itemize}


%%%%%%%%%%%%%%%%%%%%%% Chapter Betriebssystem %%%%%%%%%%%%%%%%%%%%%
\chapter{Betriebssystem}\label{Betriebssystem}

\section{Raspbian}\label{Raspbian}

\renewcommand{\kapitelautor}{Autor: Clemens Scharwitzl}

% Das ist der Text in meinem

\#Chapter\#

\begin{itemize}
\tightlist
\item
  Liste \emph{Liste }Liste *Liste
\end{itemize}


\section{Sicherheit}\label{Sicherheit}

\renewcommand{\kapitelautor}{Autor: Clemens Scharwitzl}

% Das ist der Text in meinem

\#Chapter\#

\begin{itemize}
\tightlist
\item
  Liste \emph{Liste }Liste *Liste
\end{itemize}


\section{Ola}\label{ola}

\renewcommand{\kapitelautor}{Autor: Clemens Scharwitzl}

% Das ist der Text in meinem

\#Chapter\#

\begin{itemize}
\tightlist
\item
  Liste \emph{Liste }Liste *Liste
\end{itemize}


\section{WiringPi}\label{WiringPi}

\renewcommand{\kapitelautor}{Autor: Clemens Scharwitzl}

% Das ist der Text in meinem

\#Chapter\#

\begin{itemize}
\tightlist
\item
  Liste \emph{Liste }Liste *Liste
\end{itemize}


%%%%%%%%%%%%%%%%%%%%%% Chapter Backend %%%%%%%%%%%%%%%%%%%%%
\chapter{Backend}\label{Backend}

\section{PHP Extension}\label{PHP-Extension}

\renewcommand{\kapitelautor}{Autor: Clemens Scharwitzl}

% Das ist der Text in meinem

\#Chapter\#

\begin{itemize}
\tightlist
\item
  Liste \emph{Liste }Liste *Liste
\end{itemize}


\section{Websocket}\label{Websocket}

\renewcommand{\kapitelautor}{Autor: Clemens Scharwitzl}

% Das ist der Text in meinem

\#Chapter\#

\begin{itemize}
\tightlist
\item
  Liste \emph{Liste }Liste *Liste
\end{itemize}


\section{LDAP}\label{LDAP}

\renewcommand{\kapitelautor}{Autor: Clemens Scharwitzl}

% Das ist der Text in meinem

\#Chapter\#

\begin{itemize}
\tightlist
\item
  Liste \emph{Liste }Liste *Liste
\end{itemize}


\section{DMX}\label{DMX}

\renewcommand{\kapitelautor}{Autor: Clemens Scharwitzl}

\input{markdown/Clemens.md/DMX.md}

\section{Infrarot}\label{Infrarot}

\renewcommand{\kapitelautor}{Autor: Clemens Scharwitzl}

% Das ist der Text in meinem

\#Chapter\#

\begin{itemize}
\tightlist
\item
  Liste \emph{Liste }Liste *Liste
\end{itemize}


\section{Mischpult}\label{Mischpult}

\renewcommand{\kapitelautor}{Autor: Clemens Scharwitzl}

% Das ist der Text in meinem

\#Chapter\#

\begin{itemize}
\tightlist
\item
  Liste \emph{Liste }Liste *Liste
\end{itemize}


\section{SQLite}\label{SQLite}

\renewcommand{\kapitelautor}{Autor: Clemens Scharwitzl}

% Das ist der Text in meinem

\#Chapter\#

\begin{itemize}
\tightlist
\item
  Liste \emph{Liste }Liste *Liste
\end{itemize}


%%%%%%%%%%%%%%%%%%%%%% Chapter Frontend %%%%%%%%%%%%%%%%%%%%%
\chapter{Frontend}\label{Frontend}

\section{CSS}\label{CSS}

\renewcommand{\kapitelautor}{Autor: Dominik Nußbaumer}

    \hypertarget{flexbox}{%
\subsection{Flexbox}\label{flexbox}}

Flexbox, offiziell CSS Flexible Box Layout Module, ist eine neue Art und
ein neues Konzept um eindimensionale Layouts auf Webseiten umzusetzen.
Die herkömmliche Art Objekte auf einer Webseite zu positionieren ist,
fixe Positionen und Maße zu vergeben.

Doch bei Flexbox werden bestimmte Regeln festgelegt, diese machen das
Verhalten der Webseite vorhersagbar bei einer Veränderung der
Bildschirmgröße. Anschließend ist es dem Browser überlassen, die Breite,
Höhe, Position und Anordnung zu wählen.

\hypertarget{das-konzept}{%
\paragraph{Das Konzept}\label{das-konzept}}

Die Grundidee ist es, dem Flex-Container die Möglichkeit zu geben, die
Maße der Elemente so zu verändern, dass der Platz auf unterschiedlichen
Bildschirmaufslösungen bestmöglich ausgenutzt ist. Um das zu erzielen
lässt das Elternelement die Kindelemente je nach Bedarf wachsen oder
schrumpfen.

\hypertarget{technische-spezifikation}{%
\paragraph{technische Spezifikation}\label{technische-spezifikation}}

Innerhalb eines \textless{}div\textgreater{} Tags können die einzelnen
Elemente ihre Größe ``flexibel'' verändern. Sie wachsen, um freien Platz
zu verwenden oder schrumpfen, um innerhalb des Elternobkjekts zu bleiben
und einen Overflow zu vermeiden. Der große Vorteil des Flexbox Layouts
ist die Richtungsunabhängigkeit. Dadurch ist es sehr flexibel, was
Orientierungsänderungen bei mobilen Geräten oder Auflösungsänderungen
auf Desktop Geräten betrifft.

\hypertarget{erkluxe4rung-anhand-eines-realen-beispiels}{%
\paragraph{Erklärung anhand eines realen
Beispiels}\label{erkluxe4rung-anhand-eines-realen-beispiels}}

Auf dem Dashboard soll eine seitliche Navigation angezeigt werden, die
auf mobilen Geräten an den unteren Rand des Bildschirms wandert, siehe
Abbildung 1.

\begin{figure}
\centering
\includegraphics{../../Design/Flexbox_Illustration_1.png}
\caption{alt text}
\end{figure}

Mithilfe von Flexbox ist dieses Verhalten einfach zu erzielen.\\
Ich erstelle ein Elternelement mit folgenden Eigenschaften:

\begin{verbatim}
.parent{
  display: flex;
  overflow: hidden;
}
\end{verbatim}

Die Kindelemente dieser Flexbox werden auf der horizontalen Hauptachse
ausgerichtet. Der Overflow auf der X- und Y-Achse wird ausgeblendet. Die
Navigation auf der Seite ist in folgendem Code-Block beschrieben.

Dieses Element ist durch order:1 das erste Element in der Flexbox. Der
Overflow auf der Y-Achse ist versteckt, um die Leiste zu fixieren.
Weiters werden die Elemente innerhalb vertikal und horizontal zentriert
und sind entlang der Y-Achse positioniert.

\begin{verbatim}
.side-nav{
  display: flex;
  order: 1;
  justify-content: center;
  align-items: center;
  flex-direction: column;
}
\end{verbatim}

Das Inhaltselement hat order:2 damit es neben dem ersten auf der X-Achse
positioniert wird. Ebenso ist der Overflow auf der Y-Achse versteckt.

\begin{verbatim}
.content{
  overflow-y: hidden;
  display: flex;
  justify-content: center;
  flex-direction: column;
  order: 2;
}
\end{verbatim}

Damit die Navigation auf mobilen Geräten am unteren Rand positioniert
ist, benötigen wir eine Media Query. Mithilfe dieser können CSS-Stile
anhand von verschiedenen Eigenschaften wie z.B. Bildschirmauflösung oder
Seitenverhältnis manipuliert werden. Im untenstehenden Code-Block wird
dies veranschaulicht. Indem wir die Hauptachse des Flexbox
Elternelements auf die Y-Achse ändern, werden die beiden Kindelemente
nun vertikal verteilt. Damit nun auch die Navigation unter dem Inhalt
positioniert ist ändern wir die order auf 2. Weiters müssen die Höhe und
Breite angepasst werden.

\begin{verbatim}
@media (max-width: 576px){
  .parent{
    flex-direction: column; //changed
  }

  .side-nav{
      order: 2;             //changed
      width: 100vw;         //changed
      height: 66px;         //changed
    }
  }
\end{verbatim}


\section{jQuery}\label{jQuery}

\renewcommand{\kapitelautor}{Autor: Dominik Nußbaumer}

% Das ist der Text in meinem

\#Chapter\#

\begin{itemize}
\tightlist
\item
  Liste \emph{Liste }Liste *Liste
\end{itemize}


\section{Design}\label{Design}

\renewcommand{\kapitelautor}{Autor: Dominik Nußbaumer}

% Das ist der Text in meinem

\#Chapter\#

\begin{itemize}
\tightlist
\item
  Liste \emph{Liste }Liste *Liste
\end{itemize}


%%%%%%%%%%%%%%%%%%%%%% Chapter Zusätzliches %%%%%%%%%%%%%%%%%%%%%
\chapter{Zusätzliches}\label{Zusätzlicghes}

\section{Bedienungsanleitung}\label{Bedienungsanleitung}

\renewcommand{\kapitelautor}{Autor: Clemens Scharwitzl}

% Das ist der Text in meinem

\#Chapter\#

\begin{itemize}
\tightlist
\item
  Liste \emph{Liste }Liste *Liste
\end{itemize}


\section{Beleuchtungskonzept Konferenzsaal}\label{Beleuchtungskonzept-Konferenzsaal}

\renewcommand{\kapitelautor}{Autor: Clemens Scharwitzl}

% Das ist der Text in meinem

\#Chapter\#

\begin{itemize}
\tightlist
\item
  Liste \emph{Liste }Liste *Liste
\end{itemize}


% \chapter{Ziele}

% Das erste Kapitel stellt die Ziele der DA (inkl. individuelle Ziele
% aller Mitarbeiter) da.\todo{viel Text schreiben}


% \chapter{Formatierung}

% wer hat diese Kapitel geschrieben oder leer
% \renewcommand{\kapitelautor}{Autor: Hans Huber}

% \input{text/demo_kap1.tex}


% \chapter{Planung}

% Das komplette nächste Kapitel wird in der externen Datei diplomarbeit2.tex gespeichert.
% Es wird an dieser Stelle im Dokument eingebaut.
% Damit ist es möglich, mehrere Personen an diverse Teile der Diplomarbeit arbeiten zu lassen.

% \input{markdown/diplomarbeit2.md.tex}

% \input{markdown/einstellungen.md.tex}

% \input{text/demo_fuelltext.tex}


\appendix

\chapter{Anhang 1\label{chap:Anhang-1}}

was auch immer: technische Dokumentationen etc.

Zusätzlich sollte es geben:
\begin{itemize}
\item Abkürzungsverzeichnis
\item Quellenverzeichnis (hier: Bibtex im Stil plaindin)
\end{itemize}
\printindex{}

%% Flattersatz -- damit werden die langen URLs besser umgebrochen
\raggedright %% eventuell auskommentieren
\bibliographystyle{plaindin}%Alternative unsrtdin - Nummern im Text aufsteigend
\bibliography{diplom}


\cleardoublepage
\newcommand{\Messbox}[2]{%Parameters: #1=Breite, #2=Hoehe
\setlength{\unitlength}{1.0mm}%
\begin{picture}(#1,#2)%
\linethickness{0.05mm}%
\put(0,0){\dashbox{0.2}(#1,#2)%
{\parbox{#1mm}{%
\centering\footnotesize
%{\bf MESSBOX}\\
Breite $ = #1 {\textrm\ mm}$\\
Höhe $ = #2 {\textrm\ mm}$
}}}\end{picture}
}
\begin{center} {\Large --- Druckgröße kontrollieren! ---}
\bigskip

\Messbox{100}{50} % Angabe der Breite/Hoehe in mm
\bigskip

{\Large --- Diese Seite nach dem Druck entfernen! ---}
\todo{Diese Seite nach dem Druck entfernen!}
\end{center}
\end{document}
