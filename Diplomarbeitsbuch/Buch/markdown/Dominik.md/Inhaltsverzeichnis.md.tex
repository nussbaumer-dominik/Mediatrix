\hypertarget{inhaltsverzeichnis-outline}{%
\subsection{\texorpdfstring{\# Inhaltsverzeichnis /
\emph{``outline''}}{\# Inhaltsverzeichnis / ``outline''}}\label{inhaltsverzeichnis-outline}}

\hypertarget{dominik-nuuxdfbaumer-5bi-mediatrix}{%
\subsubsection{Dominik Nußbaumer \textbar{} 5BI
\textbar{}~Mediatrix}\label{dominik-nuuxdfbaumer-5bi-mediatrix}}

\begin{itemize}
\tightlist
\item
  Aufgabenstellung
\item
  Frontend

  \begin{itemize}
  \tightlist
  \item
    CSS

    \begin{itemize}
    \tightlist
    \item
      Flexbox -\textgreater{} wo und warum habe ich es eingesetzt, wie
      funktioniert es technisch

      \begin{itemize}
      \tightlist
      \item
        technische Spezifikation
      \item
        Einsatzbereich
      \item
        Möglichkeiten
      \end{itemize}
    \item
      CSS Grid -\textgreater{} wofür es sehr gut geeignet ist

      \begin{itemize}
      \tightlist
      \item
        technische Spezifikation
      \end{itemize}
    \item
      Flexbox vs.~Grid - Was ich wann + warum

      \begin{itemize}
      \tightlist
      \item
        Gegenüberstellung
      \item
        verschiedene Einsatzmöglichkeiten
      \end{itemize}
    \item
      normalize.css

      \begin{itemize}
      \tightlist
      \item
        Funktion
      \end{itemize}
    \end{itemize}
  \item
    jQuery - warum verwende ich es -\textgreater{}

    \begin{itemize}
    \tightlist
    \item
      Zeit Sparpotenzial
    \end{itemize}
  \item
    Design

    \begin{itemize}
    \tightlist
    \item
      Produktwebsite
    \item
      Dashboard
    \item
      UX Fokus auf ``technisch unversierte'' Benutzer
    \end{itemize}
  \end{itemize}
\item
  Websockets + Clemens

  \begin{itemize}
  \tightlist
  \item
    technische Spezifikation
  \item
    Kommunikation mit dem Mischpult
  \end{itemize}
\end{itemize}
