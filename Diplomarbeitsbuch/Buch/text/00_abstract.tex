
Thats why -- the translated text ,,Kurzfassung`` (this should be
a translation).

% Im englischen Abstract sollte inhaltlich das Gleiche stehen wie in
% der deutschen Kurzfassung. Versuchen Sie daher, die Kurzfassung präzise
% umzusetzen, ohne aber dabei Wort für Wort zu übersetzen. Beachten
% Sie bei der Übersetzung, dass gewisse Redewendungen aus dem Deutschen
% im Englischen kein Pendant haben oder völlig anders formuliert werden
% müssen und dass die Satzstellung im Englischen sich (bekanntlich)
% vom Deutschen stark unterscheidet. Es empfiehlt sich übrigens – auch
% bei höchstem Vertrauen in die persönlichen Englischkenntnisse – eine
% kundige Person für das „proof reading“ zu engagieren. Die richtige
% Übersetzung für „Diplomarbeit“ ist übrigens schlicht thesis, allenfalls
% „diploma thesis“ oder „Master’s thesis“, auf keinen Fall aber „diploma
% work“ oder gar „dissertation“\citep{hagenberg}.

% Wichtig ist wegen des Abteilens ein \code{\textbackslash{}begin\{english\}}
% bzw. \code{\textbackslash{}selectlanguage\{naustrian\}}.
